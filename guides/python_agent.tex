% Options for packages loaded elsewhere
\PassOptionsToPackage{unicode}{hyperref}
\PassOptionsToPackage{hyphens}{url}
\PassOptionsToPackage{dvipsnames,svgnames,x11names}{xcolor}
%
\documentclass[
  letterpaper,
  DIV=11,
  numbers=noendperiod]{scrartcl}

\usepackage{amsmath,amssymb}
\usepackage{iftex}
\ifPDFTeX
  \usepackage[T1]{fontenc}
  \usepackage[utf8]{inputenc}
  \usepackage{textcomp} % provide euro and other symbols
\else % if luatex or xetex
  \usepackage{unicode-math}
  \defaultfontfeatures{Scale=MatchLowercase}
  \defaultfontfeatures[\rmfamily]{Ligatures=TeX,Scale=1}
\fi
\usepackage{lmodern}
\ifPDFTeX\else  
    % xetex/luatex font selection
\fi
% Use upquote if available, for straight quotes in verbatim environments
\IfFileExists{upquote.sty}{\usepackage{upquote}}{}
\IfFileExists{microtype.sty}{% use microtype if available
  \usepackage[]{microtype}
  \UseMicrotypeSet[protrusion]{basicmath} % disable protrusion for tt fonts
}{}
\makeatletter
\@ifundefined{KOMAClassName}{% if non-KOMA class
  \IfFileExists{parskip.sty}{%
    \usepackage{parskip}
  }{% else
    \setlength{\parindent}{0pt}
    \setlength{\parskip}{6pt plus 2pt minus 1pt}}
}{% if KOMA class
  \KOMAoptions{parskip=half}}
\makeatother
\usepackage{xcolor}
\setlength{\emergencystretch}{3em} % prevent overfull lines
\setcounter{secnumdepth}{-\maxdimen} % remove section numbering
% Make \paragraph and \subparagraph free-standing
\makeatletter
\ifx\paragraph\undefined\else
  \let\oldparagraph\paragraph
  \renewcommand{\paragraph}{
    \@ifstar
      \xxxParagraphStar
      \xxxParagraphNoStar
  }
  \newcommand{\xxxParagraphStar}[1]{\oldparagraph*{#1}\mbox{}}
  \newcommand{\xxxParagraphNoStar}[1]{\oldparagraph{#1}\mbox{}}
\fi
\ifx\subparagraph\undefined\else
  \let\oldsubparagraph\subparagraph
  \renewcommand{\subparagraph}{
    \@ifstar
      \xxxSubParagraphStar
      \xxxSubParagraphNoStar
  }
  \newcommand{\xxxSubParagraphStar}[1]{\oldsubparagraph*{#1}\mbox{}}
  \newcommand{\xxxSubParagraphNoStar}[1]{\oldsubparagraph{#1}\mbox{}}
\fi
\makeatother

\usepackage{color}
\usepackage{fancyvrb}
\newcommand{\VerbBar}{|}
\newcommand{\VERB}{\Verb[commandchars=\\\{\}]}
\DefineVerbatimEnvironment{Highlighting}{Verbatim}{commandchars=\\\{\}}
% Add ',fontsize=\small' for more characters per line
\usepackage{framed}
\definecolor{shadecolor}{RGB}{241,243,245}
\newenvironment{Shaded}{\begin{snugshade}}{\end{snugshade}}
\newcommand{\AlertTok}[1]{\textcolor[rgb]{0.68,0.00,0.00}{#1}}
\newcommand{\AnnotationTok}[1]{\textcolor[rgb]{0.37,0.37,0.37}{#1}}
\newcommand{\AttributeTok}[1]{\textcolor[rgb]{0.40,0.45,0.13}{#1}}
\newcommand{\BaseNTok}[1]{\textcolor[rgb]{0.68,0.00,0.00}{#1}}
\newcommand{\BuiltInTok}[1]{\textcolor[rgb]{0.00,0.23,0.31}{#1}}
\newcommand{\CharTok}[1]{\textcolor[rgb]{0.13,0.47,0.30}{#1}}
\newcommand{\CommentTok}[1]{\textcolor[rgb]{0.37,0.37,0.37}{#1}}
\newcommand{\CommentVarTok}[1]{\textcolor[rgb]{0.37,0.37,0.37}{\textit{#1}}}
\newcommand{\ConstantTok}[1]{\textcolor[rgb]{0.56,0.35,0.01}{#1}}
\newcommand{\ControlFlowTok}[1]{\textcolor[rgb]{0.00,0.23,0.31}{\textbf{#1}}}
\newcommand{\DataTypeTok}[1]{\textcolor[rgb]{0.68,0.00,0.00}{#1}}
\newcommand{\DecValTok}[1]{\textcolor[rgb]{0.68,0.00,0.00}{#1}}
\newcommand{\DocumentationTok}[1]{\textcolor[rgb]{0.37,0.37,0.37}{\textit{#1}}}
\newcommand{\ErrorTok}[1]{\textcolor[rgb]{0.68,0.00,0.00}{#1}}
\newcommand{\ExtensionTok}[1]{\textcolor[rgb]{0.00,0.23,0.31}{#1}}
\newcommand{\FloatTok}[1]{\textcolor[rgb]{0.68,0.00,0.00}{#1}}
\newcommand{\FunctionTok}[1]{\textcolor[rgb]{0.28,0.35,0.67}{#1}}
\newcommand{\ImportTok}[1]{\textcolor[rgb]{0.00,0.46,0.62}{#1}}
\newcommand{\InformationTok}[1]{\textcolor[rgb]{0.37,0.37,0.37}{#1}}
\newcommand{\KeywordTok}[1]{\textcolor[rgb]{0.00,0.23,0.31}{\textbf{#1}}}
\newcommand{\NormalTok}[1]{\textcolor[rgb]{0.00,0.23,0.31}{#1}}
\newcommand{\OperatorTok}[1]{\textcolor[rgb]{0.37,0.37,0.37}{#1}}
\newcommand{\OtherTok}[1]{\textcolor[rgb]{0.00,0.23,0.31}{#1}}
\newcommand{\PreprocessorTok}[1]{\textcolor[rgb]{0.68,0.00,0.00}{#1}}
\newcommand{\RegionMarkerTok}[1]{\textcolor[rgb]{0.00,0.23,0.31}{#1}}
\newcommand{\SpecialCharTok}[1]{\textcolor[rgb]{0.37,0.37,0.37}{#1}}
\newcommand{\SpecialStringTok}[1]{\textcolor[rgb]{0.13,0.47,0.30}{#1}}
\newcommand{\StringTok}[1]{\textcolor[rgb]{0.13,0.47,0.30}{#1}}
\newcommand{\VariableTok}[1]{\textcolor[rgb]{0.07,0.07,0.07}{#1}}
\newcommand{\VerbatimStringTok}[1]{\textcolor[rgb]{0.13,0.47,0.30}{#1}}
\newcommand{\WarningTok}[1]{\textcolor[rgb]{0.37,0.37,0.37}{\textit{#1}}}

\providecommand{\tightlist}{%
  \setlength{\itemsep}{0pt}\setlength{\parskip}{0pt}}\usepackage{longtable,booktabs,array}
\usepackage{calc} % for calculating minipage widths
% Correct order of tables after \paragraph or \subparagraph
\usepackage{etoolbox}
\makeatletter
\patchcmd\longtable{\par}{\if@noskipsec\mbox{}\fi\par}{}{}
\makeatother
% Allow footnotes in longtable head/foot
\IfFileExists{footnotehyper.sty}{\usepackage{footnotehyper}}{\usepackage{footnote}}
\makesavenoteenv{longtable}
\usepackage{graphicx}
\makeatletter
\newsavebox\pandoc@box
\newcommand*\pandocbounded[1]{% scales image to fit in text height/width
  \sbox\pandoc@box{#1}%
  \Gscale@div\@tempa{\textheight}{\dimexpr\ht\pandoc@box+\dp\pandoc@box\relax}%
  \Gscale@div\@tempb{\linewidth}{\wd\pandoc@box}%
  \ifdim\@tempb\p@<\@tempa\p@\let\@tempa\@tempb\fi% select the smaller of both
  \ifdim\@tempa\p@<\p@\scalebox{\@tempa}{\usebox\pandoc@box}%
  \else\usebox{\pandoc@box}%
  \fi%
}
% Set default figure placement to htbp
\def\fps@figure{htbp}
\makeatother

\KOMAoption{captions}{tableheading}
\makeatletter
\@ifpackageloaded{tcolorbox}{}{\usepackage[skins,breakable]{tcolorbox}}
\@ifpackageloaded{fontawesome5}{}{\usepackage{fontawesome5}}
\definecolor{quarto-callout-color}{HTML}{909090}
\definecolor{quarto-callout-note-color}{HTML}{0758E5}
\definecolor{quarto-callout-important-color}{HTML}{CC1914}
\definecolor{quarto-callout-warning-color}{HTML}{EB9113}
\definecolor{quarto-callout-tip-color}{HTML}{00A047}
\definecolor{quarto-callout-caution-color}{HTML}{FC5300}
\definecolor{quarto-callout-color-frame}{HTML}{acacac}
\definecolor{quarto-callout-note-color-frame}{HTML}{4582ec}
\definecolor{quarto-callout-important-color-frame}{HTML}{d9534f}
\definecolor{quarto-callout-warning-color-frame}{HTML}{f0ad4e}
\definecolor{quarto-callout-tip-color-frame}{HTML}{02b875}
\definecolor{quarto-callout-caution-color-frame}{HTML}{fd7e14}
\makeatother
\makeatletter
\@ifpackageloaded{caption}{}{\usepackage{caption}}
\AtBeginDocument{%
\ifdefined\contentsname
  \renewcommand*\contentsname{Table of contents}
\else
  \newcommand\contentsname{Table of contents}
\fi
\ifdefined\listfigurename
  \renewcommand*\listfigurename{List of Figures}
\else
  \newcommand\listfigurename{List of Figures}
\fi
\ifdefined\listtablename
  \renewcommand*\listtablename{List of Tables}
\else
  \newcommand\listtablename{List of Tables}
\fi
\ifdefined\figurename
  \renewcommand*\figurename{Figure}
\else
  \newcommand\figurename{Figure}
\fi
\ifdefined\tablename
  \renewcommand*\tablename{Table}
\else
  \newcommand\tablename{Table}
\fi
}
\@ifpackageloaded{float}{}{\usepackage{float}}
\floatstyle{ruled}
\@ifundefined{c@chapter}{\newfloat{codelisting}{h}{lop}}{\newfloat{codelisting}{h}{lop}[chapter]}
\floatname{codelisting}{Listing}
\newcommand*\listoflistings{\listof{codelisting}{List of Listings}}
\makeatother
\makeatletter
\makeatother
\makeatletter
\@ifpackageloaded{caption}{}{\usepackage{caption}}
\@ifpackageloaded{subcaption}{}{\usepackage{subcaption}}
\makeatother

\usepackage{bookmark}

\IfFileExists{xurl.sty}{\usepackage{xurl}}{} % add URL line breaks if available
\urlstyle{same} % disable monospaced font for URLs
\hypersetup{
  pdftitle={Python Agent},
  pdfauthor={Paolo Bosetti},
  colorlinks=true,
  linkcolor={blue},
  filecolor={Maroon},
  citecolor={Blue},
  urlcolor={Blue},
  pdfcreator={LaTeX via pandoc}}


\title{Python Agent}
\author{Paolo Bosetti}
\date{2025-06-02}

\begin{document}
\maketitle
\begin{abstract}
The \texttt{python\_agent} repo on GitHub provides a MADS agent with an
embedded python3 interpreter for developing MADS sgents in Python
\end{abstract}

\renewcommand*\contentsname{Table of contents}
{
\hypersetup{linkcolor=}
\setcounter{tocdepth}{3}
\tableofcontents
}

\section{Contents}\label{contents}

The Python3 MADS agent is available on
\url{https://github.com/MADS-net/python_agent}.

\subsection{Installing}\label{installing}

You need to have \texttt{python3} and \texttt{python3-dev} installed.
Then proceed as follows depending on your platform.

\subsubsection{UNIX}\label{unix}

\begin{Shaded}
\begin{Highlighting}[]
\ExtensionTok{python3} \AttributeTok{{-}m}\NormalTok{ venv .venv}
\BuiltInTok{source}\NormalTok{ .venv/bin/activate}
\ExtensionTok{pip}\NormalTok{ install numpy}
\CommentTok{\# also install other necessary Python libs}

\FunctionTok{cmake} \AttributeTok{{-}Bbuild} \AttributeTok{{-}DCMAKE\_INSTALL\_PREFIX}\OperatorTok{=}\StringTok{"}\VariableTok{$(}\ExtensionTok{mads} \AttributeTok{{-}p}\VariableTok{)}\StringTok{"}
\FunctionTok{cmake} \AttributeTok{{-}{-}build}\NormalTok{ build }\AttributeTok{{-}j6}
\FunctionTok{sudo}\NormalTok{ cmake }\AttributeTok{{-}{-}install}\NormalTok{ build}
\end{Highlighting}
\end{Shaded}

The above is tested on MacOS and Ubuntu 22.04.

\subsubsection{Windows}\label{windows}

Run the following from project root:

\begin{Shaded}
\begin{Highlighting}[]
\NormalTok{python }\OperatorTok{{-}}\NormalTok{m venv }\OperatorTok{.}\FunctionTok{venv}
\OperatorTok{.}\FunctionTok{venv}\NormalTok{\textbackslash{}Scripts\textbackslash{}activate}
\NormalTok{pip install numpy}
\CommentTok{\# also install other necessary Python libs}
\end{Highlighting}
\end{Shaded}

Then:

\begin{Shaded}
\begin{Highlighting}[]
\NormalTok{cmake }\OperatorTok{{-}}\NormalTok{Bbuild }\OperatorTok{{-}}\NormalTok{DCMAKE\_INSTALL\_PREFIX}\OperatorTok{=}\StringTok{"}\OperatorTok{$(}\NormalTok{mads }\OperatorTok{{-}}\NormalTok{p}\OperatorTok{)}\StringTok{"}
\NormalTok{cmake }\OperatorTok{{-}{-}}\NormalTok{build build }\OperatorTok{{-}{-}}\NormalTok{config Release}
\NormalTok{sudo cmake }\OperatorTok{{-}{-}}\NormalTok{install build}
\end{Highlighting}
\end{Shaded}

\begin{tcolorbox}[enhanced jigsaw, bottomtitle=1mm, opacitybacktitle=0.6, toprule=.15mm, breakable, colbacktitle=quarto-callout-warning-color!10!white, title=\textcolor{quarto-callout-warning-color}{\faExclamationTriangle}\hspace{0.5em}{Enable sudo on Windows}, leftrule=.75mm, toptitle=1mm, colframe=quarto-callout-warning-color-frame, bottomrule=.15mm, coltitle=black, titlerule=0mm, rightrule=.15mm, arc=.35mm, left=2mm, colback=white, opacityback=0]

For \texttt{sudo} to work on Windows, you need to enable it on
\emph{Settings \textgreater{} System \textgreater{} For Developers} and
set \emph{Enable sudo} to On.

\end{tcolorbox}

\subsection{Executing}\label{executing}

The new agent is installed as \texttt{mads-python}, so you can just type
\texttt{mads\ python\ -h} (or \texttt{mads-python\ -h} on Windows) to
know more:

\begin{Shaded}
\begin{Highlighting}[]
\OperatorTok{\textgreater{}}\NormalTok{ mads }\ExtensionTok{python} \AttributeTok{{-}h}
\ExtensionTok{python}\NormalTok{ ver. 1.2.6}

\ExtensionTok{Usage:}
  \ExtensionTok{python} \PreprocessorTok{[}\SpecialStringTok{OPTION...}\PreprocessorTok{]}

  \ExtensionTok{{-}p,} \AttributeTok{{-}{-}period}\NormalTok{ arg         Sampling period }\ErrorTok{(}\ExtensionTok{default}\NormalTok{ 100 ms}\KeywordTok{)}
  \ExtensionTok{{-}m,} \AttributeTok{{-}{-}module}\NormalTok{ arg         Python module to load}
  \ExtensionTok{{-}n,} \AttributeTok{{-}{-}name}\NormalTok{ arg           Agent name }\ErrorTok{(}\ExtensionTok{default}\NormalTok{ to }\StringTok{\textquotesingle{}python\textquotesingle{}}\KeywordTok{)}
  \ExtensionTok{{-}i,} \AttributeTok{{-}{-}agent{-}id}\NormalTok{ arg       Agent ID to be added to JSON frames}
  \ExtensionTok{{-}s,} \AttributeTok{{-}{-}settings}\NormalTok{ arg       Settings file path/URI}
  \ExtensionTok{{-}S,} \AttributeTok{{-}{-}save{-}settings}\NormalTok{ arg  Save settings to ini file}
  \ExtensionTok{{-}v,} \AttributeTok{{-}{-}version}\NormalTok{            Print version}
  \ExtensionTok{{-}h,} \AttributeTok{{-}{-}help}\NormalTok{               Print usage}
\end{Highlighting}
\end{Shaded}

Typically, to launch an agent named \texttt{python\_source}, which gets
its settings from a \texttt{python\_source} section in
\texttt{mads.ini}, and uses the Python module named \texttt{source}
defined in the \texttt{source.py} file and that runs every 100 ms, the
command is:

\begin{Shaded}
\begin{Highlighting}[]
\ExtensionTok{mads}\NormalTok{ python }\AttributeTok{{-}n}\NormalTok{ python\_source }\AttributeTok{{-}m}\NormalTok{ source }\AttributeTok{{-}p100}
\end{Highlighting}
\end{Shaded}

where:

\begin{itemize}
\tightlist
\item
  \texttt{-n\ python\_source} sets the agent name to
  \texttt{python\_source}, and gets its settings from the same section
  in the \texttt{mads.ini} file
\item
  \texttt{-m\ source} sets the Python module to \texttt{source.py},
  which is searched for in the Python modules search paths, see below
\item
  \texttt{-p100} sets the sampling period to 100 ms
\end{itemize}

\subsection{Python modules search
paths}\label{python-modules-search-paths}

The Python modules are searched for in the following folders:

\begin{itemize}
\tightlist
\item
  \texttt{./python}
\item
  \texttt{./scripts}
\item
  \texttt{../python}
\item
  \texttt{../scripts}
\item
  \texttt{../../python}
\item
  \texttt{../../scripts}
\item
  \texttt{INSTALL\_PREFIX\ +\ /python}
\item
  \texttt{INSTALL\_PREFIX\ +\ /scripts}
\end{itemize}

plus any path listed in the \texttt{mads.ini} file under the
\texttt{search\_path} key (an array or a single string).

\subsection{\texorpdfstring{The \texttt{mads.ini}
section}{The mads.ini section}}\label{the-mads.ini-section}

The following fields are typically used:

\begin{Shaded}
\begin{Highlighting}[]
\KeywordTok{[python\_source]}
\DataTypeTok{period }\OtherTok{=}\StringTok{ }\DecValTok{200}
\DataTypeTok{venv }\OtherTok{=}\StringTok{ "/path/to/.venv"}
\DataTypeTok{python\_module }\OtherTok{=}\StringTok{ "my\_source"}
\DataTypeTok{search\_paths }\OtherTok{=}\StringTok{ ["/path/to/python/folder"}
\end{Highlighting}
\end{Shaded}

\begin{tcolorbox}[enhanced jigsaw, bottomtitle=1mm, opacitybacktitle=0.6, toprule=.15mm, breakable, colbacktitle=quarto-callout-warning-color!10!white, title=\textcolor{quarto-callout-warning-color}{\faExclamationTriangle}\hspace{0.5em}{Warning}, leftrule=.75mm, toptitle=1mm, colframe=quarto-callout-warning-color-frame, bottomrule=.15mm, coltitle=black, titlerule=0mm, rightrule=.15mm, arc=.35mm, left=2mm, colback=white, opacityback=0]

The section name must match the \texttt{-m} option argument when you
launch the agent, so in the case aboxe you must use
\texttt{-m\ python\_source}.

\end{tcolorbox}

\subsection{Module Types}\label{module-types}

Python modules can be of type \texttt{source}, \texttt{filter}, or
\texttt{sink}. The module type is defined by setting a top level
variable like this, typically at the beginning of the script, just after
the various \texttt{import}s:

\begin{Shaded}
\begin{Highlighting}[]
\NormalTok{agent\_type }\OperatorTok{=} \StringTok{"sink"}
\end{Highlighting}
\end{Shaded}

All the modules \textbf{must} implement a \texttt{setup()} function,
which is expected to use the dictionary available in the module variable
\texttt{params} (a dictionary)~to do initial setup (opening ports or
files, etc.)

\textbf{Source} modules \textbf{must} implement a \texttt{get\_output()}
function, that produces the JSON string that will be published.

\textbf{Filter} modules \textbf{must} implement a \texttt{process()}
function, that is supposed to operate on the last received data
dictionary (available as \texttt{data}, a module variable) and produce a
JSON string that will be published.

\textbf{Sink} modules \textbf{must} implement a
\texttt{deal\_with\_data()} function, that operates on the \texttt{data}
dictionary, a module variable.

\section{Examples}\label{examples}

\begin{tcolorbox}[enhanced jigsaw, bottomtitle=1mm, opacitybacktitle=0.6, toprule=.15mm, breakable, colbacktitle=quarto-callout-note-color!10!white, title=\textcolor{quarto-callout-note-color}{\faInfo}\hspace{0.5em}{Note}, leftrule=.75mm, toptitle=1mm, colframe=quarto-callout-note-color-frame, bottomrule=.15mm, coltitle=black, titlerule=0mm, rightrule=.15mm, arc=.35mm, left=2mm, colback=white, opacityback=0]

To be completed

\end{tcolorbox}

\begin{center}\rule{0.5\linewidth}{0.5pt}\end{center}




\end{document}
